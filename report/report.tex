\documentclass[a4paper,10pt]{article}
\usepackage{libertine}
\usepackage[utf8]{inputenc}

\usepackage[french]{babel}
\usepackage[autolanguage]{numprint}
\usepackage{amsmath}
\usepackage{xcolor}
\usepackage{graphicx}
\usepackage{hyperref}
\usepackage{float}
\graphicspath{{./img/}}
\DeclareGraphicsExtensions{.png, .jpeg, .jpg}
\renewcommand*\thesection{\arabic{section}}
\hypersetup{
    colorlinks,
    citecolor=blue,
    filecolor=blue,
    linkcolor=blue,
    urlcolor=blue
}

\usepackage{geometry}
%\geometry{scale=0.82, nohead}
\usepackage{listings}
\usepackage{caption}
\usepackage{subcaption}
\lstset{keywordstyle=\color{blue}}
\lstset{stringstyle=\color{brown}}
\lstset{showspaces=false}
\lstset{showtabs=false}
\lstset{extendedchars=true}
\lstset{columns=flexible}
\lstset{keepspaces=true}
\lstset{numbers=left, numberstyle=\tiny, stepnumber=1, numbersep=5pt}


\usepackage{tikz}

\title{Calcul efficace des plus longs préfixes communs}
\author{ Rémi \textsc{Bois} et Loïc \textsc{Jankowiac}}
\date{\today}
\begin{document}

\maketitle


\section{Introduction}
\label{sec:intro}


\subsection{Contexte}
\label{sec:context}

\subsection{Intérêt}
\label{sec:interest}



\section{Une structure permettant de retrouver les lcps}
\label{sec:heightstruct}


\subsection{Le tableau Height}
\label{sec:struct}



\subsection{L'algorithme}
\label{sec:algo}



\section{Applications}
\label{sec:appli}

\subsection{La compression Block Sorting}
\label{sec:blocksorting}

\subsection{Simulation d'un parcours bottom-up}
\label{sec:bottomup}

\section{Conclusion}
\label{sec:conclusion}





\end{document}