\documentclass[10pt]{beamer}

\usetheme{Warsaw}
\beamertemplatenavigationsymbolsempty

\usepackage[utf8]{inputenc}
\usepackage[francais]{babel}
\usepackage{hyperref}
\usepackage{amsmath}
\usepackage{graphicx}
\graphicspath{{./img/}}
\DeclareGraphicsExtensions{.png, .jpeg, .jpg}


\renewcommand*\thesection{\arabic{section}}


\AtBeginSection[]{%
  \begin{frame}<beamer>
    \frametitle{Plan}
    \tableofcontents[sectionstyle=show/hide,subsectionstyle=hide/show/hide]
  \end{frame}
  \addtocounter{framenumber}{-1}
}



\title{Obtention efficace des Longest Common Prefixes}
\author{Rémi Bois, Loïc Jankowiac}
\date{\today}

\begin{document}

\begin{frame}
  \maketitle

\end{frame}

\begin{frame}
  \tableofcontents
\end{frame}

\section{Contexte}
\label{sec:context}



\begin{frame}
  \frametitle{Recherche de pattern à partir d'un tableau des suffixes
    et des longest common prefixes}
  %description de l'efficacité de la recherche via le tableau des
  %suffixes en termes de complexité algorithmique et spaciale
\end{frame}

\begin{frame}
  \frametitle{Quelques notations}
  %lcp, A, et tout ce dont on aura besoin dans sec:algo
\end{frame}

\begin{frame}
  \frametitle{Un élément manquant : les longest common prefixes}
  %cadre dans lequel on se place : on a le texte et le tableau des
  %suffixes. Il manque les lcp pour faire une recherche efficace
\end{frame}


\begin{frame}
  \frametitle{Un contexte réaliste ?}
  %intro sur les deux problèmes qu'on explorera plus tard
\end{frame}

\section{Calculer les lcps en O(n)}
\label{sec:algo}

%Faut voir comment on organise ça. Faut le faire tourner sur un
%exemple (abraca ?)


\section{Application à la recherche de pattern dans un texte compressé}
\label{sec:appcompress}

\begin{frame}
  \frametitle{La compression par Block-Sorting}
  %complexité, implémentations
\end{frame}

\begin{frame}
  \frametitle{Comment ça marche ?}
  %exemple sur abraca. On s'arrête après la rotation et l'obtention de
  %L, I. On explique "grossièrement" la suite (en une dizaine de secondes)
\end{frame}

\begin{frame}
  \frametitle{Pourquoi ça nous intéresse ?}
  %on peut retrouver le tableau des suffixes à partir de L, I mais il
  %nous manque les informations sur les LCPs. On peut les retrouver en O(n).
\end{frame}

%Une conclusion à cette section ?


\section{Simulation d'un parcours bottom-up de l'arbre des suffixes}
\label{sec:appbottomup}

%Aucune idée pour l'instant. Pas sûr qu'on aura le temps de présenter
%en détail cette section.

\section{Conclusion}
\label{sec:conclusion}

\begin{frame}
  \frametitle{Un algorithme performant et simple}
  %rappel de la complexité, des cas d'utilisations, ...
\end{frame}

\begin{frame}
  \frametitle{Des questions ?}
  %une image sympa
\end{frame}

\begin{frame}
  \frametitle{Références}
  %les articles cités
\end{frame}

\end{document}