\documentclass[10pt]{beamer}

\usetheme{Warsaw}
\beamertemplatenavigationsymbolsempty

\usepackage[utf8x]{inputenc}
\usepackage[francais]{babel}
\usepackage{hyperref}
\usepackage{amsmath}
\usepackage{graphicx}
\usepackage{tikz}
\usepackage{multicol}
\usetikzlibrary{automata,positioning}
\graphicspath{{./img/}}
\DeclareGraphicsExtensions{.png, .jpeg, .jpg}


\renewcommand*\thesection{\arabic{section}}


\AtBeginSection[]{%
  \begin{frame}<beamer>
    \frametitle{Plan}
    \tableofcontents[sectionstyle=show/hide,subsectionstyle=hide/show/hide]
  \end{frame}
  \addtocounter{framenumber}{-1}
}



\title{Obtention efficace des Longest Common Prefixes}
\author{Rémi Bois, Loïc Jankowiak}
\date{\today}

\begin{document}

\begin{frame}
  \maketitle

\end{frame}

\begin{frame}
  \tableofcontents
\end{frame}

\section{Contexte}
\label{sec:context}


\begin{frame}
  \frametitle{Recherche de motif à partir d'un tableau des suffixes
    et des longest common prefixes}
  %description de l'efficacité de la recherche via le tableau des
  %suffixes en termes de complexité algorithmique et spaciale

  \begin{block}{arbre des suffixes vs tableau des suffixes} 
    Le tableau des suffixes est une représentation compacte de l'arbre
    des suffixes. Le principal avantage du tableau des suffixes comparé à
    l'arbre des suffixes est la place mémoire occupée, environ 4 fois
    moins importante. Cette contrainte est importante lors de
    recherches sur de longs textes, notamment en bioinformatique \cite{Raffinot11}.
  \end{block}

  \pause

  \begin{block}{La recherche de motifs}
    Le tableau des suffixes permet de trouver les occurrences d'un
    motif dans un texte en $O(m \times log(n))$. Associé aux informations sur
    les plus longs préfixes communs (Longest Common Prefixes,
    \emph{lcp}), il permet une recherche en $O(m + log(n))$ \cite{Manber93}.
  \end{block}
  
\end{frame}

\begin{frame}
  \frametitle{Un élément manquant : les longest common prefixes}
  %cadre dans lequel on se place : on a le texte et le tableau des
  %suffixes. Il manque les lcp pour faire une recherche efficace

  \begin{block}{Les lcps pendant la construction}
      Dans les faits, le calcul des lcp se fait lors de la construction de
      l'arbre des suffixes, ou du tableau des suffixes. Il ne rajoute pas
      de complexité à ces constructions (l'ordre de grandeur reste
      identique).
  \end{block}
  \pause
  \begin{block}{Les lcps après la construction}
    L'article présenté propose un moyen efficace d'obtenir
    les informations des lcp lorsque celles-ci ne sont pas
    disponibles \cite{Kasai01}.
  \end{block}

\end{frame}


\begin{frame}
  \frametitle{Un contexte réaliste ?}
  %intro sur les deux problèmes qu'on explorera plus tard
  
  \begin{block}{Une application concrète}
    Un algorithme de compression basé sur la transformation de
    Burrow-Wheeler \cite{Burrows94}, très utilisé (\emph{bizp2}), permet
    de retrouver, lors de la décompression, le tableau des suffixes. 
    
    Il ne manque alors plus que les informations sur les lcps pour faire
    une recherche efficace. 
  \end{block}

\end{frame}

\section{Calculer les lcps en O(n)}
\label{sec:algo}

%Faut voir comment on organise ça. Faut le faire tourner sur un
%exemple (abraca ?)

\begin{frame}
  \frametitle{Quelques notations}

  Le calcul des \textit{lcp} utilise les notations suivantes~:
  \begin{itemize}
  \item \textit{lcp} : Longest Common Prefixes
  \item \textit{A} : Un texte
  \item \textit{Pos} : le tableau des suffixes de A trié par ordre
    lexicographique.
  \item \textit{Rank} : le tableau inverse de \textit{Pos}~:
    \begin{center}
    $\mathit{Rank}[\mathit{Pos}[i]] = i$
    \end{center}
  \item \textit{Height} : le tableau des \textit{lcp}. Il correspond à la
    hauteur du plus petit ancêtre commun de deux feuilles consécutives dans
    l'arbre des suffixes~:
    \begin{center}
    $\mathit{Height}[i] = \mathit{lcp}(A_{\mathit{Pos}[i-1]},
      A_{\mathit{Pos}[i]})$
    \end{center}
  \end{itemize}
  \hfill \\ \hfill \\

  L'idée est d'itérer sur les suffixes triés par indice de début, et de
  calculer les \textit{lcp} des suffixes adjacents dans \textit{Pos}.
\end{frame}

\begin{frame}
  \frametitle{Algorithme GetHeight}

  \begin{block}{Propriété 1}
  Le \textit{lcp} de deux sous-chaînes est le minimum des \textit{lcp} de
  toutes les sous-chaînes adjacentes de l'intervalle dans le tableau
  \textit{Pos}.
  \end{block}
  \hfill \\

  \begin{columns}
    \begin{column}{.48\textwidth}
    \scriptsize
    \begin{tabular}{lc}
      $A_{i}$ & $\mathit{lcp}(A_{\mathit{Pos}[i-1]}, A_{\mathit{Pos}[i]})$\\
      \hline
      \texttt{4 abbca\$}                      & \\
      \texttt{1 abcabbca\$}                   & 2\\
      \texttt{8 a\$}                          & 1\\
      \texttt{\textcolor{blue}{5 bbca\$}}     & 0\\
      \texttt{\textcolor{blue}{2 bcabbca\$}}  & \textcolor{blue}{1}\\
      \texttt{\textcolor{blue}{6 bca\$}}      & \textcolor{blue}{3}\\
      \texttt{3 cabbca\$}                     & 0\\
      \texttt{7 ca\$}                         & 2\\
      \texttt{9 \$}                           & 0\\
    \end{tabular}
    \normalsize
    \end{column}
    \begin{column}{.48\textwidth}
    $\mathit{lcp}(A_{5}, A_{6}) = \mathit{min}(1, 3) = 1$\\ \hfill \\

    $\mathit{lcp}(A_{1}, A_{2}) = \mathit{min}(1, 0, 1) = 0$
    \end{column}
  \end{columns}
\end{frame}

\begin{frame}
  \frametitle{Algorithme GetHeight}

  \begin{block}{Propriété 2}
  Lorsque l'on supprime le premier caractère du suffixe (ie. lorsque l'on
  avance d'un caractère), l'ordre des suffixes comparés est conservé dans
  \textit{Pos}.
  \end{block}
  \hfill \\

  \texttt{\ldots\\
  \textcolor{blue}{abaa\$}\\
  \textcolor{blue}{abcabaa\$}\\
  \textcolor{red}{baa\$}\\
  \textcolor{red}{bcabaa\$}\\
  \ldots}\\
\end{frame}

\begin{frame}

  \frametitle{Algorithme GetHeight}
  
  
  \begin{block}{Propriété 3}
    Lorsque l'on avance d'un caractère, le \textit{lcp} diminue de 1
    si le \textit{lcp} précédent était $\geq$ 1.
  \end{block}
  
  \pause

  \begin{columns}
    \begin{column}{.48\textwidth}
      
      Aucun caractère ne diffère :
      
      aaaaa\$

      aaaa\$

      $\hookrightarrow$ lcp = 4

      aaaa\$

      aaa\$

      $\hookrightarrow$ lcp = 3

      
      Le lcp diminue de 1 à chaque fois qu'on passe au suffixe suivant.
      
      % Left Part
    \end{column}%
    \hfill%
    \pause
    \begin{column}{.48\textwidth}
        
      Au moins un caractère diffère (mais lcp $>$ 1):
      
        aabaa\$

        abaa\$

        $\hookrightarrow$ lcp = 1

        abaa\$

        baa\$

        $\hookrightarrow$ lcp = 0
        
        Le lcp diminue de 1 à chaque fois qu'on passe au suffixe suivant.
        
      \end{column}%
      \end{columns}
  
\end{frame}

% \begin{frame}
%   \frametitle{Pourquoi calculer \textit{Height[q]} à partir de
%     \textit{Height[p]}~?}
%
%   On pose $p = \mathit{Rank}[i-1]$ et $q = \mathit{Rank}[i]$.
%   \begin{block}{Théorème}
%   Si $\mathit{Height}[p] > 1$, alors~:
%   $\mathit{Height}[q] ≥ \mathit{Height}[p] - 1$
%   \end{block}
%   \begin{block}{Preuve}
%
%   \end{block}
% \end{frame}


\begin{frame}
  \frametitle{Algorithme GetHeight}
  \scriptsize
  \begin{columns}
    \column{.5\textwidth}
      GetHeight\\
      Entrée: Une chaîne A et son tableau des suffixes\\ \hfill \\
      \begin{tabular}{r|l}
        \texttt{1}  & \verb!pour i de 1 à n faire!\\
        \texttt{2}  & \verb! ~Rank[Pos[i]] ← i!\\
        \texttt{3}  & \verb!fin pour!\\
        \texttt{4}  & \verb!h ← 0!\\
        \texttt{5}  & \verb!pour i de 1 à n faire!\\
        \texttt{6}  & \verb! ~si Rank[i] > 1 alors!\\
        \texttt{7}  & \verb! ~ ~k ← Pos[Rank[i]-1]!\\
        \texttt{8}  & \verb! ~ ~tantque A[i+h] = A[k+h] faire!\\
        \texttt{9}  & \verb! ~ ~ ~h ← h+1!\\
        \texttt{10} & \verb! ~ ~fin tq!\\
        \texttt{11} & \verb! ~ ~Height[Rank[i]] ← h!\\
        \texttt{12} & \verb! ~ ~si h > 0 alors h ← h-1 fin si!\\
        \texttt{13} & \verb! ~fin si!\\
        \texttt{14} & \verb!fin pour!\\
      \end{tabular}\\
    \column{.5\textwidth}
    \begin{minipage}[c][.6\textheight][c]{\linewidth}
      \pause
      \begin{tabular}{ll}
        Suffixes & Pos\\
        \hline
        \verb!1 atalata\$! & \texttt{\alert<3,5>{3 alata\$}}\\
        \verb!2 talata\$!  & \texttt{\alert<3,7>{1 atalata\$}}\\
        \verb!3 alata\$!   & \texttt{\alert<7>{5 ata\$}}\\
        \verb!4 lata\$!    & \texttt{\alert<6>{7 a\$}}\\
        \verb!5 ata\$!     & \texttt{\alert<4,6>{4 lata\$}}\\
        \verb!6 ta\$!      & \texttt{\alert<4>{2 talata\$}}\\
        \verb!7 a\$!       & \texttt{6 ta\$}\\
        \verb!8 \$!        & \texttt{8 \$}\\
      \end{tabular}\\
      \hfill \\ \hfill \\

      \begin{tabular}{ll}
        $i$      & \texttt{\alert<5>{1}\alert<3>{2}\alert<7>{3}4\alert<6>{5}\alert<4>{6}78}\\
        \hline
        chaîne & \verb!atalata\$!\\
        \texttt{Pos}    & \texttt{\alert<3,5>{3}\alert<3,7>{1}\alert<7>{5}\alert<6>{7}\alert<4,6>{4}\alert<4>{2}68}\\
        \texttt{Rank}   & \texttt{\alert<3>{2}\alert<4>{6}\alert<5>{1}\alert<6>{5}\alert<7>{3}748}\\
        \texttt{Height} & \only<2>{\texttt{........}
                          }\only<3>{\texttt{.1......}
                          }\only<4-5>{\texttt{.1...0..}
                          }\only<6>{\texttt{.1..00..}
                          }\only<7>{\texttt{.13.00..}
                          }\only<8->{\texttt{.1310020}}\\
      \end{tabular}\\
      \hfill \\ \hfill \\
      \only<-2>{\\ \\ \\ \\ \\ \\
      }\only<3>{h ← 0\\
      \alert<3>{i = 1} ; Rank[i] = 2 ; k ← 3\\
      ~ ~ ~ A[1] = A[3] ? Oui $\Rightarrow$ h ← 1\\
      ~ ~ ~ A[2] = A[4] ? Non $\Rightarrow$ Height[2] ← 1\\
      ~ ~ ~ h ← 0\\
      \\
      }\only<4>{h = 0\\
      \alert<4>{i = 2} ; Rank[i] = 6 ; k ← 4\\
      ~ ~ ~ A[2] = A[4] ? Non $\Rightarrow$ Height[6] ← 0\\
      \\ \\ \\
      }\only<5>{h = 0\\
      \alert<5>{i = 3} ; Rank[i] = 1\\
      \\ \\ \\ \\
      }\only<6>{h = 0\\
      \alert<6>{i = 4} ; Rank[i] = 5 ; k ← 7\\
      ~ ~ ~ A[4] = A[7] ? Non $\Rightarrow$ Height[5] ← 0\\
      \\ \\ \\
      }\only<7>{h = 0\\
      \alert<7>{i = 5} ; Rank[5] = 3 ; k ← 1\\
      ~ ~ ~ A[5] = A[1] ? Oui $\Rightarrow$ h ← 1\\
      ~ ~ ~ A[6] = A[2] ? Oui $\Rightarrow$ h ← 2\\
      ~ ~ ~ A[7] = A[3] ? Oui $\Rightarrow$ h ← 3\\
      ~ ~ ~ A[8] = A[4] ? Non $\Rightarrow$ Height[3] ← 3\\
      ~ ~ ~ h ← 2
      }\only<8>{\\ \ldots \\ \\ \\ \\ \\}

    \end{minipage}
  \end{columns}
  \normalsize
\end{frame}

\begin{frame}
  \frametitle{Algorithme GetHeight}
  \begin{figure}
    \includegraphics[width=0.6\textwidth]{sufftree_atalata}
  \end{figure}
  \hfill \\
  \begin{center}
    \texttt{Height : . 1 3 1 0 0 2 0}
  \end{center}
\end{frame}

\section{Application à la recherche de motif dans un texte compressé}
\label{sec:appcompress}

\begin{frame}
  \frametitle{La compression par Block-Sorting}
  %complexité, implémentations

  \begin{block}{Un bon compromis}
    Block-sorting parvient à obtenir un très bon taux de compression
    tout en étant rapide (vitesse proche de LZ et compression proche
    des modèles statistiques \cite{Burrows94}).
  \end{block}

  \begin{block}{Basée sur une idée ``simple''}
    Le but est de rapprocher les caractères identiques d'un texte et
    d'utiliser un encodage ``move-to-front'' consistant à indiquer la
    distance du prochain caractère identique. On utilise ensuite un
    codage de Huffman pour encoder la suite obtenue.
  \end{block}


\end{frame}

\begin{frame}
  \frametitle{Comment ça marche ?}
  %exemple sur abcabbca$. On s'arrête après la rotation et l'obtention de
  %L, I. On explique "grossièrement" la suite (en une dizaine de
  %secondes)
  \begin{figure}
    \includegraphics[width=0.22\textwidth]{start_burrows}
  \end{figure}

\end{frame}

\begin{frame}
  \frametitle{Comment ça marche ?}
  %exemple sur abcabbca$. On s'arrête après la rotation et l'obtention de
  %L, I. On explique "grossièrement" la suite (en une dizaine de
  %secondes)
  \begin{figure}
    \includegraphics[width=0.21\textwidth]{1_burrows}
  \end{figure}

\end{frame}

\begin{frame}
  \frametitle{Comment ça marche ?}
  %exemple sur abcabbca$. On s'arrête après la rotation et l'obtention de
  %L, I. On explique "grossièrement" la suite (en une dizaine de
  %secondes)
  \begin{figure}
    \includegraphics[width=0.2\textwidth]{2_burrows}
  \end{figure}

\end{frame}

\begin{frame}
  \frametitle{Comment ça marche ?}
  %exemple sur abcabbca$. On s'arrête après la rotation et l'obtention de
  %L, I. On explique "grossièrement" la suite (en une dizaine de
  %secondes)
  \begin{figure}
    \includegraphics[width=0.2\textwidth]{3_burrows}
  \end{figure}

\end{frame}

\begin{frame}
  \frametitle{Comment ça marche ?}
  %exemple sur abcabbca$. On s'arrête après la rotation et l'obtention de
  %L, I. On explique "grossièrement" la suite (en une dizaine de
  %secondes)
  \begin{figure}
    \includegraphics[width=0.5\textwidth]{full_burrows}
  \end{figure}

\end{frame}

\begin{frame}
  \frametitle{Pourquoi ça nous intéresse ?}
  %on peut retrouver le tableau des suffixes à partir de L, I mais il
  %nous manque les informations sur les LCPs. On peut les retrouver en
  %O(n).

  \begin{block}{Comme un air de famille entre I et Pos}
    Le tableau I est en fait le tableau des suffixes.
    Il ne manque donc plus que les informations sur les lcp pour faire
    une recherche efficace.
  \end{block}

\end{frame}

%Une conclusion à cette section ?


\section{Simulation d'un parcours bottom-up de l'arbre des suffixes}
\label{sec:appbottomup}


\begin{frame}
  \frametitle{Parcours bottom-up ? Késako ?}

  \begin{block}{Formellement}
    Il s'agit de parcourir d'abord les feuilles les plus à gauche,
    puis de partir vers la droite tout en remontant vers la racine. On
    peut le définir comme le parcours des ``branches les plus à
    droites'' (Rightmost branches).
  \end{block}

  \begin{block}{Avec les mains}
      \begin{figure}
    \includegraphics[width=0.6\textwidth]{sufftree_atalata}
  \end{figure}
  \end{block}

\end{frame}



\begin{frame}

  \frametitle{Le parcours en question}

    \scriptsize
  \begin{columns}
    \column{.5\textwidth}
      TraverseWithArray\\
      Entrée: Le tableau des suffixes Pos et le tableau Height\\ \hfill \\
      \begin{tabular}{r|l}
        \texttt{1}  & \verb!S:=(-1,-1); n:=|T| /*Init stack S*/!\\
        \texttt{2}  & \verb!for k:=1 to n+1 do /*k-th stage*/!\\
        \texttt{3}  & \verb! ~(Llca, Hlca) := (k-1, Height[k]);!\\
        \texttt{4}  & \verb! ~(L, H) := top(S);!\\
        \texttt{5}  & \verb! ~while (H > Hlca) do!\\
        \texttt{6}  & \verb! ~ ~(L, H) := pop(S), R := k-1;!\\
        \texttt{7}  & \verb! ~ ~report triple (L, R, H);!\\
        \texttt{8}  & \verb! ~ ~Llca := L;/*Update boundary*/!\\
        \texttt{9}  & \verb! ~ ~(L, H) := top(S);!\\
        \texttt{10} & \verb! ~od!\\
        \texttt{11} & \verb! ~if (H < Hlca) then!\\
        \texttt{12} & \verb! ~ ~Push((Llca,Hlca), S); fi;!\\
        \texttt{13} & \verb! ~Push((k, n - Pos[k] + 1), S);!\\
        \texttt{14} & \verb!od!\\
      \end{tabular}\\
    \column{.5\textwidth}
       Pos : 3 1 5 7 4 2 6 8\\
    Height : . 1 3 1 0 0 2 0\\
    \vspace{0.5cm}
    \pause
    k=1\\
    ~ (Llca, Hlca) = (0,0)\\
    ~ (L, H) = (-1, -1)\\
    ~ $H > Hlca$ ? Non; $H < Hlca$ ? Oui → push(0,0)\\ 
    ~ push(1,6) \\
    stack : (-1,-1) , (0,0), (1,6)\\
    \vspace{1cm}
    \pause
    
    k=2\\
    ~ (Llca, Hlca) = (1,1)\\
    ~ (L, H) = (1, 6)\\
    ~ $H > Hlca$ ? Oui\\
    ~ ~ \textcolor{red}{report(1,1,6) (Left, Right, Lenght)}\\
    ~ ~ Llca = 1\\
    ~ ~ (L, H) = (0,0)\\
    ~ $H < Hlca$ ? Oui → push(1,1)\\
    ~ push(2,8)\\
    stack : (-1,-1) , (0, 0), (1,1), (2,8)\\


  \end{columns}

\end{frame}

\begin{frame}
  \frametitle{Le parcours en question II}
  \scriptsize
  \begin{columns}
    \column{.5\textwidth}
      TraverseWithArray\\
      Entrée: Le tableau des suffixes Pos et le tableau Height\\ \hfill \\
      \begin{tabular}{r|l}
        \texttt{1}  & \verb!S:=(-1,-1); n:=|T| /*Init stack S*/!\\
        \texttt{2}  & \verb!for k:=1 to n+1 do /*k-th stage*/!\\
        \texttt{3}  & \verb! ~(Llca, Hlca) := (k-1, Height[k]);!\\
        \texttt{4}  & \verb! ~(L, H) := top(S);!\\
        \texttt{5}  & \verb! ~while (H > Hlca) do!\\
        \texttt{6}  & \verb! ~ ~(L, H) := pop(S), R := k-1;!\\
        \texttt{7}  & \verb! ~ ~report triple (L, R, H);!\\
        \texttt{8}  & \verb! ~ ~Llca := L;/*Update boundary*/!\\
        \texttt{9}  & \verb! ~ ~(L, H) := top(S);!\\
        \texttt{10} & \verb! ~od!\\
        \texttt{11} & \verb! ~if (H < Hlca) then!\\
        \texttt{12} & \verb! ~ ~Push((Llca,Hlca), S); fi;!\\
        \texttt{13} & \verb! ~Push((k, n - Pos[k] + 1), S);!\\
        \texttt{14} & \verb!od!\\
      \end{tabular}\\
    \column{.5\textwidth}

       Pos : 3 1 5 7 4 2 6 8\\
    Height : . 1 3 1 0 0 2 0\\
    \vspace{0.5cm}

    Et on continue (en raccourci)\\
    k=3\\
    ~ \textcolor{red}{report(2,2,8)}\\
    ~ push(2,3) ; push(3,4)\\
    stack : (-1,-1) , (0,0), (2,3), (3,4)\\
    \vspace{1cm}
    \pause
    
    k=4\\
    ~ \textcolor{red}{report(3,3,4)}\\
    ~ \textcolor{red}{report(2,3,3)} (premier noeud interne)\\
    ~ push(2,1) ; push(4,2)\\
    stack : (-1,-1) , (0,0), (2,1), (4,2)\\

  \end{columns}


\end{frame}

\begin{frame}
  \frametitle{Bilan du Bottom-up}

  \begin{block}{Un parcours efficace}
    Parcours de l'arbre en O(n)
    Les parcours bottom-up peuvent résoudrent de nombreux problèmes
    (répétitions en tandem, plus longue sous chaîne commune,
    décomposition de Zip Lempel).
  \end{block}

  \begin{block}{Un algorithme extensible}
    Abouelhoda et al ont montré qu'on pouvait étendre cet algorithme
    pour effectuer des parcours top-down \cite{Abouelhoda200453}.
  \end{block}

\end{frame}


%Aucune idée pour l'instant. Pas sûr qu'on aura le temps de présenter
%en détail cette section.

\section{Conclusion}
\label{sec:conclusion}

\begin{frame}
  \frametitle{Un algorithme performant et simple}
  %rappel de la complexité, des cas d'utilisations, ...

  \begin{block}{Complexité}
    \begin{itemize}
    \item Un tableau de taille n-1 à stocker (même ordre de grandeur que les
      tableaux de lcp classiques)
    \item Un espace mémoire de $13n $ lors du calcul (peut être réduit à $9n$ \cite{Manzini04})
    \item Un calcul en O(n)
    \end{itemize}
  \end{block}

  \pause

  \begin{block}{Des applications concrètes}
    \begin{itemize}
    \item Recherche après décompression d'un texte sous Burrows-Wheeler
    \item Peut simuler un parcours bottom-up de l'arbre des suffixes
    \end{itemize}
  \end{block}

  \pause

  \begin{block}{Une structure extensible}
    Abouelhoda et al \cite{Abouelhoda200453} ont proposé d'améliorer la structure obtenue par
    un arbre d'intervalles des lcps qui permet d'obtenir la liste des
    enfants d'un noeud interne à l'arbre.
  \end{block}

\end{frame}

\begin{frame}
  \frametitle{Des questions ?}
  %une image sympa

\begin{tikzpicture}[shorten >=1pt,node distance=2cm,on grid,auto] 
   \node[state,initial] (q_0)   {$q_0$}; 
   \node[state] (q_1) [above right=of q_0] {$q_1$}; 
   \node[state] (q_2) [below right=of q_0] {$q_2$}; 
   \node[state] (q_5) [right=3cm of q_0]{$q_5$};
   \node[state] (q_6) [right=2.5cm of q_5]{$q_6$};
   \node[state,accepting](q_7) [right=3cm of q_6] {$q_7$};
    \path[->] 
    (q_0) edge  node {Avez} (q_1)
    edge  node [swap] {Posez} (q_2)
    (q_1) edge node {vous} (q_5)
    (q_2) edge node [swap] {nous} (q_5)
    (q_5) edge node {des} (q_6)
    (q_6) edge node {questions} (q_7);
\end{tikzpicture}


\end{frame}

\begin{frame}[allowframebreaks]
  \frametitle{Références}
  %les articles cités
  \bibliographystyle{amsalpha}
  \small
  \bibliography{./pres.bib}
\end{frame}

\end{document}